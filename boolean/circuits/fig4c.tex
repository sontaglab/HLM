\documentclass[border=3mm]{standalone}
\usepackage[rgb]{xcolor}
\usepackage{tikz}
\usetikzlibrary{arrows,shapes.gates.logic.US,shapes.gates.logic.IEC,calc}


\pgfmathdeclarerandomlist{MyRandomColors}{
	{black} 
	{blue}
	{brown}
	{cyan}
	{darkgray}
	{gray}
	{green}
	{lightgray}
	{lime}
	{magenta}
	{olive}
	{orange}
	{pink}
	{purple}
	{red}
	{teal}
	{violet}
	{white}
	{yellow}
}

\begin{document}
	\thispagestyle{empty}
	\tikzstyle{branch}=[fill,shape=circle,minimum size=3pt,inner sep=0pt]
	
	\def\prunelist#1{%
		\expandafter\edef\csname pgfmath@randomlist@#1\endcsname
		{\the\numexpr\csname pgfmath@randomlist@#1\endcsname-1\relax}
		\count@\pgfmath@randomtemp 
		\loop
		\expandafter\let
		\csname pgfmath@randomlist@#1@\the\count@\expandafter\endcsname
		\csname pgfmath@randomlist@#1@\the\numexpr\count@+1\relax\endcsname
		\ifnum\count@<\csname pgfmath@randomlist@#1\endcsname\relax
		\advance\count@\@ne
		\repeat}
	

	\begin{tikzpicture}[label distance=2mm]
	
	
	\node (x0) at (0,1) {$x_1$};
	\node (x2) at (1,0) {$x_3$};
	\pgfmathsetseed{128}

	\pgfmathrandomitem{\RandomColor}{MyRandomColors} 
	\node[not gate US, draw, logic gate inputs=nn, fill=\RandomColor] at ($(x0)+(1,0)$) (Nor1) {};
	\pgfmathrandomitem{\RandomColor}{MyRandomColors} 
	\node[nor gate US, draw, logic gate inputs=nn, fill=\RandomColor] at ($(Nor1)+(1,-0.5)$) (Nor2) {};
	\pgfmathrandomitem{\RandomColor}{MyRandomColors} 
	\node[not gate US, draw, logic gate inputs=nn, fill=\RandomColor] at ($(Nor2)+(1,0)$) (Nor3) {};
	
	\draw (x0) |- (Nor1.input);
	\draw (Nor1.output) |- (Nor2.input 1);
	\draw (x2) |- (Nor2.input 2);
	\draw (Nor2.output) |- (Nor3.input);
	\node (QS1) at ($(Nor3)+(1,0)$){$QS$};
	\draw (Nor3.output) |- (QS1);
	
	\end{tikzpicture}
\end{document} 